\documentclass[]{kyvernitis-resume}
\fullname{Zhiping WANG}
\usepackage{amsmath}
\usepackage{bm}
\usepackage{fontspec}
% \renewcommand{\rmdefault}{qcs} % 设置文档的默认字体为 Quicksand
% \renewcommand{\sfdefault}{qcs} % 设置 sans-serif 字体为 Quicksand



% \jobtitle{Sofware Engineer}
\usepackage{geometry}
\geometry{a4paper,left=0.9cm,right=0.9cm}
\geometry{top = 1.3cm,bottom=1cm}

\usepackage{xcolor}
\usepackage{hyperref}
\begin{document}
\resumeheader
{\ORCiD{0009-0004-9890-9809}}
{\emailperson{w17611688963@gmail.com}}
{\emailacademic{zhpwang818@gmail.com}}
% {\phone{+86 18311151638}}
{\website{wang-zhiping.github.io}}
{\phone{+86 18311151638}}
{\nationality{Chinese}}
% {\github{example}}
% {\linkedin{zhiping-wang-21857a256}}
{}
{}
{}

% \vspace{-0.3cm} 
\begin{section}{\Large Education}
    \begin{subsectionnobullet}{School of Physical Science and Technology, Lanzhou University (Project 985)}{One of China's top 10 science universities}{Lanzhou, China}{Sep 2020 -- Present}
    \vspace{1mm}
        \italicitem{\textbf{Major}: Physics (in {National Training Base for Research and Teaching Talents in Basic Science Disciplines)}\\Bachelor of Science degree expected in July 2024}\vspace{1mm}
        \italicitem{\textbf{GPA}: 85.25/100,\hspace{5mm} \textbf{Ranking}: 4/20 (20 Selected from 52)
}
        % \italicitem{Ranking:4/20 in Physics Base Class}
        \vspace{2mm}
        \italicitem{\textbf{Major courses:}Fourier optics(89), Computational Physics (100), Methods of Mathematical Physics II (99), Optoelectronic Technology and Applications(94), AI and Big Data(97),  Theoretical Mechanics, Statistical Physics, Electrodynamics, Quantum Mechanics, Ferro Magnetism, Magnetic Materials and Measurements, Linear Algebra.}
    \end{subsectionnobullet}
\end{section}

\sectiontable{\Large Honors and Awards}{
    \entry{Excellent Bachelor's Thesis}{ \hfill \textit{Jun. 2024}}
    \entry{Outstanding Student Scholarship}{ \hfill \textit{Sep. 2023 and Sep. 2022}}
    \entry{China Undergraduate Physics Tournament(Northwest Region)}{Second Prize%(As a team leader)
    \hfill \textit{Jul. 2022}}
    \entry{China Undergraduate Physics Tournament(Northwest Region)}{First Prize%(As a contestant) 
    \hfill \textit{Jun. 2021}}
}

\begin{section}{\Large Publication}
\begin{itemize}
    
    \item[{[1]}] \underline{Zhiping Wang}, Tianci Feng, and An Pan. \emph{\textbf{Fusion-Based Enhancement of Multi-Exposure Fourier ptychographic microscopy}}.
     \begin{itemize}
         \item[*] The project's result can be found at the \href{https://wang-zhiping.github.io/ResearchSummary&Reporting/FPM/fusion.html}{project link}.
    \item Submitted to the journal \emph{\textbf{APL Photonics}}.
    \end{itemize}

    \item[{[1]}] Fannuo Xu $\,^{\dag}$, \underline{Zhiping Wang $\,^{\dag}$}, Zipei Wu, Houyou Lai, Yizheng Liao, and An Pan. \emph{\textbf{Slicing-free, wide-field quantitative phase imaging via feature-domain Fourier ptychographic microscopy}}.

    \begin{itemize}
        % \item[*] The project's result can be found at the \href{https://wang-zhiping.github.io/ResearchSummary&Reporting/FPM/fusion.html}{project link}.
   \item Submitted to the journal \emph{{Optics Letters}} (Manuscript ID531347) and under peer review.
   \end{itemize}


 \item[{[3]}] Tianci Feng, Aiye Wang, \underline{Zhiping Wang}, Yizheng Liao, and An Pan. \emph{\textbf{A Linear-Space-Variant Model for Fourier Ptychographic Microscopy}}.
    \begin{itemize}
    \item Proposed linear space-variant FPM model, which better matches the raw data to reduce global artifacts.
    \item \textbf{Accepted} and produced by \emph{\textbf{Optics Letters}} in \href{https://opg.optica.org/ol/abstract.cfm?doi=10.1364/OL.522745}{Link}. (DOI:\href{http://dx.doi.org/10.1364/OL.522745 }{10.1364/OL.522745})


    \end{itemize}

\item[{[4]}] Fannuo Xu, Zipei Wu, Chao Tan, Yizheng Liao, \underline{Zhiping Wang}, Keru Chen, and An Pan. \emph{\textbf{Ten Years On: A Review of Fourier Ptychographic Microscopy}}.
\begin{itemize}
\item \textbf{Accepted} by \emph{Cells} on February 8, 2024, accessible via the following \href{https://www.mdpi.com/2073-4409/13/4/324}{link}.(DOI:\href{hhttps://doi.org/10.3390/cells13040324}{10.3390/cells13040324})
    \end{itemize}



    \item \underline{Zhiping Wang}. \emph{\textbf{Performance of Coherent Ising Machine on Weighted NP-hard Problem}}.
    \begin{itemize}
    \item[*] The project's code and result can be found at the \href{https://github.com/Wang-Zhiping/Exploring-the-Performance-of-Coherent-Ising-Machine-in-weighted-NP-Hard-Problems}{GitHub project link}.
    \item Preprint
    \end{itemize}




    \item \underline{Zhiping Wang}, \textbf{Bachelor’s Thesis}: Exploring Advancements in  Slicing-free Fourier Ptychographic Microscopy.
    \begin{itemize}
        \item[*] Instructor: Dr. Hao Jia (Lanzhou University and KAUST) and Dr. An Pan (Chinese Academy of Sciences)
        \item Summarized some of my work on Fourier Ptychographic Microscopy. 
        \item Achieved an \textbf{excellent} rating for my thesis through \textbf{oral defense}.
        \end{itemize}

\end{itemize}
\end{section}


\newpage

% \vspace{-0.2cm} 
\begin{section}{\Large Research/Projects Experience}
% \vspace{0.2cm}
% {\small \textcolor{orange}{Here are several representative ongoing or completed research. For more information, please visit my }\href{https://wang-zhiping.github.io/}{\textcolor{blue}{personal website}}\textcolor{orange}{.}}



    \begin{subsection}{Research on Fast Fourier Ptychographic Based on Illumination Control}{Research Internship, Supervisor: Dr. An Pan, Pioneering Interdiscipline Center \footnote{One of the interdisciplinary research centers within the \textbf{national laboratories}, among China's \textbf{top four} optical research bases.}
 of Chinese Academy of Sciences}{Aug 2023--present}{}
    % % \begin{itemize}
     \item[*] The project's result can be found at the \href{https://wang-zhiping.github.io/ResearchSummary&Reporting/FPM/single-fast.html}{GitHub project link1} and  \href{https://wang-zhiping.github.io/ResearchSummary&Reporting/FPM/fusion.html}{GitHub project link2}.
        \item Studied articles related to the principles of Fourier Ptychographic Microscopy and actively participated in experiments to gain insights into the details.
        \item Performed numerical simulations to assess the effect of various led on image restoration, explored relevant literature and theory to seek support for reducing overlap rates;  experiment still in the planning.
        \item Successfully implemented rapid imaging on a miniaturized system using the new algorithm, simultaneously expanded, with the potential to cross-link with image fusion techniques for enhanced phase recovery.
        % \item 

    % \end{itemize}
    
    \end{subsection}

    % \begin{subsection}{Improving The Efficiency of Photoacoustic Conversion in Photoacoustic Imaging}{Supervisor Dr. Fei Gao, School of Information and Engineering, Shanghai Tech}{Mar 2023--present}{Shanghai, China}
    % % % \begin{itemize}
    %     \item xxx
    %     \item xxx

    % \end{itemize}
    % \end{subsection}
    
    
    \begin{subsection}{Exploring the Performance of Coherent Ising Machine in weighted NP-Hard Problems}{Independent Study, Advisor: Jie Zhu, School of ECE, Purdue University}{Dec 2022--Aug 2023}{}
    % % \begin{itemize}
    \item[*] The Project's code and details can be viewed at the  \href{https://github.com/Wang-Zhiping/Exploring-the-Performance-of-Coherent-Ising-Machine-in-weighted-NP-Hard-Problems}{GitHub project link}.
    
        \item Replicated prior research using an Optical Parametric Oscillators (OPO)-based coherent Ising machine for numerical simulations, utilizing theoretical equations, and applying the Runge-Kutta method to solve differential equations in Python.
        \item Utilized coherent Ising machine to address number partitioning problems and MAX-CUT in unweighted graphs, for the MAX-CUT problem, the success possibility of the Ising machine approach was higher.
        \item Applied the MaxCut problem to weighted graphs and found similar trends, suggesting that the success possibility might be associated with the weights.
    % \end{itemize}
    \end{subsection}

    
    \begin{subsection}{Reproduction of Reverse Design of Nano-Optical Structures By Neural Networks}{Research Assistant, Advisor: Dr. Hao Jia, Lanzhou University \& KAUST}{Apr 2022--Mar 2023}{}
    % % \begin{itemize}
        \item Carried out literature research on the reverse design methods for optoelectronics devices and their applications.
        \item Created an optical system employing a tandem architecture that combines forward modeling and inverse design based on the work of Yu Zongfu’s team.
        \item Coded in Python using TensorFlow to capture the trends mentioned in the paper using a small sample dataset.
        % Additionally, real-world data was incorporated for testing, achieving an accuracy rate of approximately 99\% %.
    % \end{itemize}
    \end{subsection}
    
% \begin{subsection}{Reproduction of the Work on Stochastic Resonance and Mean Time to Extinction}{Advisors: Dr. Zhixi Wu \& Jianyue Guan, School of Physics and Technology}{Nov. 2021}{}
%         \item[*] The project's code and details can be found at the \href{https://github.com/Wang-Zhiping/reproduced-the-work-of-the-article-PhysRevE.104.024133}{GitHub project link}.
%         \item Learned and reproduced the theories of stochastic resonance and other physical principles.
%         \item Coded to simulate fluctuations in biological populations, successfully replicating all aspects outlined in the paper.
%         \item Successfully studied and reproduced all the work within two weeks, earning a perfect score upon submission as a classroom assignment (a historical first).
% \end{subsection}


    % \begin{subsection}{Palm print identification}{Advisor: Dr. Jizhao Liu, School of Information Science and Engineering}{Nov. 2021 – Mar. 2022}{}
    % % % \begin{itemize}
    %     \item[*] The Project's code and details can be viewed at the \href{https://github.com/Wang-Zhiping/palmprint-recognition}{GitHub project link}.
    %     \item Learned common operations in machine learning and digital image processing, especially biometrics.
    %     \item used various filters for feature extraction, and finally PCA is used for dimensionality reduction.
    %     \item Coded in Python using TensorFlow to capture the trends mentioned in the paper using a small sample dataset.
    %     \item Achieved a high success rate for the Hong Kong Polytechnic University open source database and part of the students' data recognition success
    % \end{subsection}
    %         \begin{subsection}{Exploration of the nature and causes of candle flame oscillator}{Advisor: Dr. Ning Huang, School of Physical Science and Technology }{Oct. 2020 – Sep. 2021}{}
    % % \begin{itemize}

    %     \item Searched the article to understand the conditions and root causes of the phenomenon of coupled flame oscillations when several candles are burned in close proximity to each other.

    %     \item Captured images and temperatures of the coupled oscillations and processed them with OpenCV to obtain theoretically relevant experimental data.
        
    %     \item Modeled and interpreted using thermodynamic fluid dynamics, and drawing conclusions.

    %     \item Participated in China Undergraduate Physics Tournament(Northwest Region) as a tournament topic and won the first prize in June 2021.
    % \end{subsection}
    {\small \textcolor{orange}{Here are some representative (not all) research projects. For more information, please visit my }\href{https://wang-zhiping.github.io/}{\textcolor{blue}{personal website}}\textcolor{orange}{.}}






\end{section}


% \begin{section}{Publications}
%     \begin{subsection}{Performance of Coherent Ising Machine on weighted NP-hard Problems}{}{Apr 2022-Mar 2023}{}
%     % % \begin{itemize}
%         \item Carried out literature research on the reverse design methods for optoelectronics devices and their applications.
%         \item Carried out literature research on the reverse design methods for optoelectronics devices and their applications.
%         \item Coded in Python using TensorFlow to capture the trends mentioned in the paper using a small sample dataset.
%     % \end{itemize}
%     \end{subsection}
% \end{section}

% \vspace{-0.2cm} 
\sectiontable{\Large Skills}{
    \entry{Programming:}{
       Proficient in C/C++, MATLAB, Mathematica, Python (TensorFlow, OpenCV, etc.), \LaTeX/Tex
    }
    \entry{Software:}{
        Familiar with Comsol, SolidWorks, Zemax, PixInsight; Proficient in Adobe Illustrator, Photoshop
    }
    \entry{Computing Skills:}{
        Experienced in supercomputing environments for high-performance computing tasks
        Competent in Linux for system administration and scripting
        Familiar with CUDA for GPU-accelerated computing
    }
}






\begin{section}{\Large Teaching Experience}
    \begin{subsection}{School of Physical Science and Technology, Lanzhou University}{Teaching Assistant for the Computational Physics Class}{Lanzhou, China}{September 2021 -- January 2022}
    \item Reviewed and graded student assignments, provided constructive feedback to students, and helped teachers with ongoing evaluation.
    \item Assisted students with course material, answered questions during regular office hours I held or in the class, and conducted supplemental study sessions to enhance students' understanding of complex topics.
      \item Collaborated with the course instructor to develop educational materials, including presentations and assignments, to improve the overall learning experience.
    \end{subsection}
\end{section}




% \begin{section}{Coursework}
% \begin{subsectionnobullet}{Advanced Procrastination Techniques}{Professor Delayed Gratification}{Fall 2020}{Couch, Dormroom}
%     \item{Learned advanced techniques to postpone tasks until the last possible moment. Topics included strategic distractions, perfectionism-induced paralysis, and the art of convincing yourself that you work best under pressure.}
% \end{subsectionnobullet}

% \begin{subsectionnobullet}{Mastering the Art of Excuse-Making}{Professor Imaginary Circumstances}{Fall 2019}{Office, Dean}
%     \item{Explored the art of crafting creative and believable excuses for missed deadlines and unfinished work. Special emphasis was given to the power of plausible deniability, feigning technical difficulties, and perfecting the "dog ate my homework" alibi.}
% \end{subsectionnobullet}

% \begin{subsectionnobullet}{The Science of Last-Minute Panic}{Professor Urgency Enthusiast}{Spring 2019}{Chair, Dormroom}
%     \item{Examined the psychological and physiological effects of last-minute panic on productivity. Studied the adrenaline-fueled rush of imminent deadlines, the relationship between panic and creativity, and effective strategies for harnessing the power of urgency.}
% \end{subsectionnobullet}

% \end{section}

% \sectiontable{Technical skills}{
%     \entry{Programming Languages}{JDSL, BobX, BrainF*ck, ColdFusion, Hibernate, HQL}
% }

% \sectiontable{Soft skills}{
%     \entry{Procrastination}{Procrastination at an expert level}
%     \entry{Avoiding Responsibility}{Outstanding ability to avoid meetings and responsibilities}
%     \entry{Sarcasm}{Fluent in sarcasm and irony}
% }

% \sectiontable{Awards}{
%     \entry{World's Best Napper}{International Association of Snoozers \hfill \textit{2022}}
%     \entry{Most Creative Excuses}{Academy of Procrastinators \hfill \textit{2021}}
%     \entry{Gold Medal in Avoiding Responsibilities}{Olympics of Slackers \hfill \textit{2020}}
% }

\end{document}
